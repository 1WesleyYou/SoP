% Overcoming social, economic, educational or physical barriers 
\documentclass{article}
\usepackage[T1]{fontenc}
\usepackage[utf8]{inputenc}
\usepackage{parskip}

\setlength{\parskip}{10pt}
% \setlength{\parindent}{10pt}

%  TODO: 注意这里的用词要符合论文风格, 专业名词这些

% Linespread command allows you to change line spacing for the entire document
\linespread{1.18}

% Tweak page margins
\addtolength{\oddsidemargin}{-.875in}
\addtolength{\evensidemargin}{-.875in}
\addtolength{\textwidth}{1.75in}

\addtolength{\topmargin}{-.875in}
\addtolength{\textheight}{1.75in}

\usepackage{natbib}
\usepackage{hyperref}
\usepackage{xcolor}
\usepackage{xspace}
\usepackage{fancyhdr}
\hypersetup{
    colorlinks,
    linkcolor={red!50!black},
    citecolor={blue!50!black},
    urlcolor={blue!80!black}
}

\newcommand{\HRule}{\rule{\linewidth}{0.5mm}}
\newcommand{\Hrule}{\rule{\linewidth}{0.3mm}}

% Project specific macros
\newcommand{\graphite}{GRAPHITE\xspace}
\newcommand{\wave}{WAVE\xspace}

% School specific macros
\newcommand{\schoolShort}{UMD\xspace}
\newcommand{\school}{University of Maryland\xspace}
\newcommand{\schoolLong}{University of Maryland College Park\xspace}

\newcommand{\profOne}{Prof. Ryan Huang\xspace}
\newcommand{\profTwo}{Prof. Xiaonan Huang\xspace}
\newcommand{\profThree}{Prof. Yutong Ban\xspace}

% Creates header for each page
\usepackage{fancyhdr}
\pagestyle{fancy}
\fancyhf{}
\fancyhead[LE,RO]{\header\hskip\linepagesep\vfootline\thepage}
\newskip\linepagesep \linepagesep 5pt\relax
\def\vfootline{%
    \begingroup
    	\rule[-10pt]{0.75pt}{25pt}
    \endgroup
}
\def\header{%
	\begin{minipage}[]{120pt}
		\hfill Yuchen You
    	\par \hfill 				% Formatting boilerplate
    	CS, PhD, Fall 2026 			% Area, Program, Cycle, Year
    \end{minipage}
}
\fancyhead[RE,LO]{Personal History Statement | \schoolLong}
\renewcommand\headrulewidth{0pt}

\begin{document}

\section*{Overcoming Educational Barriers}

One important test of my persistence came in EECS~482 (Introduction to Operating Systems). I initially treated it like any other demanding course, completing the projects without fully understanding the underlying ideas, and as a result I scored well below the class average on the first midterm. For a while I felt genuinely discouraged and wondered if I simply was not suited for systems.

Instead of quietly giving up, I met with the instructor, Prof.\ Ryan Huang, who encouraged me to treat the course as research training: study a real operating systems textbook, revisit each concept carefully, and use the projects as a sandbox to ask ``what if'' questions and try small extensions. Following his advice, I changed my approach---systematically reviewing the material, designing and reasoning about potential optimizations, and regularly going to office hours to discuss design trade-offs rather than just fix bugs. My performance in the second half of the course improved sharply, and I finished with a final exam score well above the average. More importantly, I learned that when I hit a wall, I can respond not by lowering my ambitions, but by redesigning my strategy and seeking out help. This mindset---combining persistence with a willingness to change my approach---is one I will carry into graduate-level research, where confusion and setbacks are a natural part of making progress.

\end{document}