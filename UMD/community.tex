% Community involvement and/or service 
\documentclass{article}
\usepackage[T1]{fontenc}
\usepackage[utf8]{inputenc}
\usepackage{parskip}

\setlength{\parskip}{10pt}
% \setlength{\parindent}{10pt}

%  TODO: 注意这里的用词要符合论文风格, 专业名词这些

% Linespread command allows you to change line spacing for the entire document
\linespread{1.18}

% Tweak page margins
\addtolength{\oddsidemargin}{-.875in}
\addtolength{\evensidemargin}{-.875in}
\addtolength{\textwidth}{1.75in}

\addtolength{\topmargin}{-.875in}
\addtolength{\textheight}{1.75in}

\usepackage{natbib}
\usepackage{hyperref}
\usepackage{xcolor}
\usepackage{xspace}
\usepackage{fancyhdr}
\hypersetup{
    colorlinks,
    linkcolor={red!50!black},
    citecolor={blue!50!black},
    urlcolor={blue!80!black}
}

\newcommand{\HRule}{\rule{\linewidth}{0.5mm}}
\newcommand{\Hrule}{\rule{\linewidth}{0.3mm}}

% Project specific macros
\newcommand{\graphite}{GRAPHITE\xspace}
\newcommand{\wave}{WAVE\xspace}

% School specific macros
\newcommand{\schoolShort}{UMD\xspace}
\newcommand{\school}{University of Maryland\xspace}
\newcommand{\schoolLong}{University of Maryland College Park\xspace}

\newcommand{\profOne}{Prof. Ryan Huang\xspace}
\newcommand{\profTwo}{Prof. Xiaonan Huang\xspace}
\newcommand{\profThree}{Prof. Yutong Ban\xspace}

% Creates header for each page
\usepackage{fancyhdr}
\pagestyle{fancy}
\fancyhf{}
\fancyhead[LE,RO]{\header\hskip\linepagesep\vfootline\thepage}
\newskip\linepagesep \linepagesep 5pt\relax
\def\vfootline{%
    \begingroup
    	\rule[-10pt]{0.75pt}{25pt}
    \endgroup
}
\def\header{%
	\begin{minipage}[]{120pt}
		\hfill Yuchen You
    	\par \hfill 				% Formatting boilerplate
    	CS, PhD, Fall 2026 			% Area, Program, Cycle, Year
    \end{minipage}
}
\fancyhead[RE,LO]{Personal History Statement | \schoolLong}
\renewcommand\headrulewidth{0pt}

\begin{document}

\section*{Community Involvement and Service}

One of the most meaningful forms of community involvement I have experienced was volunteering at a senior community center in Shanghai. Many of the older adults there had become dependent on smartphone-based public services-hospital appointment systems, transit and payment apps, health codes—but felt anxious about using them. Some had been turned away from in-person service counters and now faced a system that seemed to assume everyone was fluent with touchscreens and mini-programs.

Our volunteer group set a simple goal: help each participant complete one essential task on their own phone. I would sit beside one person at a time, ask what they needed to do that week, and then walk through the process step by step: unlocking the phone, finding the correct WeChat mini-program, navigating a dense menu, confirming a reservation. We repeated the same sequence multiple times, gradually shifting from ``let me do this for you'' to ``you try while I watch.'' When something went wrong---a mis-tap, a confusing pop-up---I treated it as an opportunity to slow down, explain what had happened, and show how to recover safely.

Over repeated visits, I watched participants move from apologizing for ``being bad with technology'' to completing tasks with only a brief prompt. Some began to stand up during sessions and demonstrate the steps to their neighbors. That transformation-from passive recipients of help to active helpers-showed me how much impact patient, one-on-one technical support can have on someone's autonomy and confidence.

This experience has shaped how I think about my role as a technologist. It reinforced that community service is not just about donating time, but about sharing technical knowledge in ways that reduce, rather than deepen, digital exclusion-an ethic I hope to carry into graduate school through outreach and accessible systems work.

\end{document}