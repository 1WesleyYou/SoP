% Engagement in leadership roles, facilitating change, or mentoring others 
\documentclass{article}
\usepackage[T1]{fontenc}
\usepackage[utf8]{inputenc}
\usepackage{parskip}

\setlength{\parskip}{10pt}
% \setlength{\parindent}{10pt}

%  TODO: 注意这里的用词要符合论文风格, 专业名词这些

% Linespread command allows you to change line spacing for the entire document
\linespread{1.18}

% Tweak page margins
\addtolength{\oddsidemargin}{-.875in}
\addtolength{\evensidemargin}{-.875in}
\addtolength{\textwidth}{1.75in}

\addtolength{\topmargin}{-.875in}
\addtolength{\textheight}{1.75in}

\usepackage{natbib}
\usepackage{hyperref}
\usepackage{xcolor}
\usepackage{xspace}
\usepackage{fancyhdr}
\hypersetup{
    colorlinks,
    linkcolor={red!50!black},
    citecolor={blue!50!black},
    urlcolor={blue!80!black}
}

\newcommand{\HRule}{\rule{\linewidth}{0.5mm}}
\newcommand{\Hrule}{\rule{\linewidth}{0.3mm}}

% Project specific macros
\newcommand{\graphite}{GRAPHITE\xspace}
\newcommand{\wave}{WAVE\xspace}

% School specific macros
\newcommand{\schoolShort}{UMD\xspace}
\newcommand{\school}{University of Maryland\xspace}
\newcommand{\schoolLong}{University of Maryland College Park\xspace}

\newcommand{\profOne}{Prof. Ryan Huang\xspace}
\newcommand{\profTwo}{Prof. Xiaonan Huang\xspace}
\newcommand{\profThree}{Prof. Yutong Ban\xspace}

% Creates header for each page
\usepackage{fancyhdr}
\pagestyle{fancy}
\fancyhf{}
\fancyhead[LE,RO]{\header\hskip\linepagesep\vfootline\thepage}
\newskip\linepagesep \linepagesep 5pt\relax
\def\vfootline{%
    \begingroup
    	\rule[-10pt]{0.75pt}{25pt}
    \endgroup
}
\def\header{%
	\begin{minipage}[]{120pt}
		\hfill Yuchen You
    	\par \hfill 				% Formatting boilerplate
    	CS, PhD, Fall 2026 			% Area, Program, Cycle, Year
    \end{minipage}
}
\fancyhead[RE,LO]{Personal History Statement | \schoolLong}
\renewcommand\headrulewidth{0pt}

\begin{document}

\section*{Leadership, Mentoring, and Facilitating Change}

One of the most important ways I have prepared for a PhD program is through mentoring as a teaching assistant for an introductory software engineering course at Shanghai Jiao Tong University. The course required teams to build a web-based game, and students entered with very different levels of programming experience. Some had written small programs before; others were seeing event-driven logic and collaborative development tools for the first time.

Early in the term, I noticed a pattern: in many groups, the most confident student quickly took over the core gameplay logic, while quieter or less-prepared teammates drifted toward peripheral roles. Over time, those students began describing themselves as “not good at coding,” and their willingness to engage with the technical work declined.

I started to treat mentoring not as simply answering questions, but as deliberately reshaping team dynamics. In discussion sessions, I introduced structured role rotation—driver, reviewer, tester—so that every student worked directly with the codebase. I designed short check-in prompts requiring each member to explain one function, data structure, or design choice. During office hours, I focused on students who were hesitant to speak in class, helping them turn vague confusion into concrete debugging steps and showing how to reason about unexpected game behavior.

By the end of the course, several students who had avoided coding at the beginning were implementing key game mechanics and proposing design improvements. This experience strengthened skills that are central to succeeding in a systems-focused PhD program: diagnosing conceptual gaps quickly, communicating at multiple levels of abstraction, and building structures in which everyone can contribute meaningfully. At Maryland, I hope to apply these mentoring habits in research groups and would welcome the opportunity to serve as a TA and support junior students in the classroom as well.

\end{document}