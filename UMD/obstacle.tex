% Overcoming social, economic, educational or physical barriers 
\documentclass{article}
\usepackage[T1]{fontenc}
\usepackage[utf8]{inputenc}
\usepackage{parskip}

\setlength{\parskip}{10pt}
% \setlength{\parindent}{10pt}

%  TODO: 注意这里的用词要符合论文风格, 专业名词这些

% Linespread command allows you to change line spacing for the entire document
\linespread{1.18}

% Tweak page margins
\addtolength{\oddsidemargin}{-.875in}
\addtolength{\evensidemargin}{-.875in}
\addtolength{\textwidth}{1.75in}

\addtolength{\topmargin}{-.875in}
\addtolength{\textheight}{1.75in}

\usepackage{natbib}
\usepackage{hyperref}
\usepackage{xcolor}
\usepackage{xspace}
\usepackage{fancyhdr}
\hypersetup{
    colorlinks,
    linkcolor={red!50!black},
    citecolor={blue!50!black},
    urlcolor={blue!80!black}
}

\newcommand{\HRule}{\rule{\linewidth}{0.5mm}}
\newcommand{\Hrule}{\rule{\linewidth}{0.3mm}}

% Project specific macros
\newcommand{\graphite}{GRAPHITE\xspace}
\newcommand{\wave}{WAVE\xspace}

% School specific macros
\newcommand{\schoolShort}{UMD\xspace}
\newcommand{\school}{University of Maryland\xspace}
\newcommand{\schoolLong}{University of Maryland College Park\xspace}

\newcommand{\profOne}{Prof. Ryan Huang\xspace}
\newcommand{\profTwo}{Prof. Xiaonan Huang\xspace}
\newcommand{\profThree}{Prof. Yutong Ban\xspace}

% Creates header for each page
\usepackage{fancyhdr}
\pagestyle{fancy}
\fancyhf{}
\fancyhead[LE,RO]{\header\hskip\linepagesep\vfootline\thepage}
\newskip\linepagesep \linepagesep 5pt\relax
\def\vfootline{%
    \begingroup
    	\rule[-10pt]{0.75pt}{25pt}
    \endgroup
}
\def\header{%
	\begin{minipage}[]{120pt}
		\hfill Yuchen You
    	\par \hfill 				% Formatting boilerplate
    	CS, PhD, Fall 2026 			% Area, Program, Cycle, Year
    \end{minipage}
}
\fancyhead[RE,LO]{Personal History Statement | \schoolLong}
\renewcommand\headrulewidth{0pt}

\begin{document}

\section*{Overcoming Educational Barriers}

The most significant barrier I have faced has been educational rather than financial or physical. For most of my schooling, success meant preparing for high-stakes exams and following clear grading rubrics. That training gave me discipline, but it did not teach me how to learn when expectations were implicit and progress depended on self-directed exploration. Entering research universities at Shanghai Jiao Tong and the University of Michigan, I had to navigate a very different academic culture; office hours, open-ended readings, and projects that were meant to be starting points rather than end goals.

This mismatch became most visible in EECS 482 (Introduction to Operating Systems). I treated the course like any other demanding class, completing the projects without fully understanding the underlying ideas, and scored well below the class average on the first midterm. For a period, I wondered if I simply was not suited for systems. Talking with the instructor, Prof. Ryan Huang, helped me see that the real barrier was not ability, but my approach: I was still acting as if there were a hidden answer key I had failed to find.

I decided to redesign my strategy. I worked through a full operating systems textbook, revisited each lecture with handwritten summaries, and started using projects as a sandbox to ask ``what if'' questions and try small extensions. I went to office hours not just to fix bugs, but to discuss design trade-offs and performance bottlenecks. My performance in the second half of the course improved sharply, but more importantly, I learned to treat confusion as a signal to change tactics and seek out resources, rather than a verdict on my potential.

This shift---from ``work harder on the same path'' to ``step back, rethink, and ask for help''---is exactly the mindset I will rely on in graduate research, where ambiguity, setbacks, and incomplete guidance are the norm.

\end{document}