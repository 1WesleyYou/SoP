% Engagement in leadership roles, facilitating change, or mentoring others 
\documentclass{article}
\usepackage[T1]{fontenc}
\usepackage[utf8]{inputenc}
\usepackage{parskip}

\setlength{\parskip}{10pt}
% \setlength{\parindent}{10pt}

%  TODO: 注意这里的用词要符合论文风格, 专业名词这些

% Linespread command allows you to change line spacing for the entire document
\linespread{1.18}

% Tweak page margins
\addtolength{\oddsidemargin}{-.875in}
\addtolength{\evensidemargin}{-.875in}
\addtolength{\textwidth}{1.75in}

\addtolength{\topmargin}{-.875in}
\addtolength{\textheight}{1.75in}

\usepackage{natbib}
\usepackage{hyperref}
\usepackage{xcolor}
\usepackage{xspace}
\usepackage{fancyhdr}
\hypersetup{
    colorlinks,
    linkcolor={red!50!black},
    citecolor={blue!50!black},
    urlcolor={blue!80!black}
}

\newcommand{\HRule}{\rule{\linewidth}{0.5mm}}
\newcommand{\Hrule}{\rule{\linewidth}{0.3mm}}

% Project specific macros
\newcommand{\graphite}{GRAPHITE\xspace}
\newcommand{\wave}{WAVE\xspace}

% School specific macros
\newcommand{\schoolShort}{UMD\xspace}
\newcommand{\school}{University of Maryland\xspace}
\newcommand{\schoolLong}{University of Maryland College Park\xspace}

\newcommand{\profOne}{Prof. Ryan Huang\xspace}
\newcommand{\profTwo}{Prof. Xiaonan Huang\xspace}
\newcommand{\profThree}{Prof. Yutong Ban\xspace}

% Creates header for each page
\usepackage{fancyhdr}
\pagestyle{fancy}
\fancyhf{}
\fancyhead[LE,RO]{\header\hskip\linepagesep\vfootline\thepage}
\newskip\linepagesep \linepagesep 5pt\relax
\def\vfootline{%
    \begingroup
    	\rule[-10pt]{0.75pt}{25pt}
    \endgroup
}
\def\header{%
	\begin{minipage}[]{120pt}
		\hfill Yuchen You
    	\par \hfill 				% Formatting boilerplate
    	CS, PhD, Fall 2026 			% Area, Program, Cycle, Year
    \end{minipage}
}
\fancyhead[RE,LO]{Personal History Statement | \schoolLong}
\renewcommand\headrulewidth{0pt}

\begin{document}

\section*{Leadership, Mentoring, and Facilitating Change}

These concerns became concrete for me when I served as a teaching assistant for ENGR~1000J, an introductory software engineering course for first-year students at Shanghai Jiao Tong University. Many of my students had never built a real project before, and our final assignment asked them to design and implement a web-based game in teams. Early in the semester, I noticed a familiar pattern: a few confident students would quickly take over the ``core'' tasks---game mechanics, backend logic---while others quietly drifted to the margins, doing only documentation or minor UI tweaks.

In labs and office hours, I made a deliberate effort to break this pattern: I worked with each team to decompose the project into smaller, meaningful components, encouraged them to rotate roles, and asked the more experienced members to explain their design choices rather than simply ``doing it for everyone.'' I spent extra time with students who were hesitant to code, helping them implement their first interactive features and debug their own work instead of handing them a finished solution. By the end of the course, many teams reported that every member had at least one cool feature they were proud to ``own,'' and several students told me that this was the first time they felt genuinely included in a technical project. Seeing first-year students gain confidence, find their voice in team discussions, and experience the satisfaction of making their ideas work on screen reinforced my belief that thoughtful mentoring and inclusive project design can significantly lower the barriers to engaging with engineering. As a Ph.D.\ student, I hope to apply the same approach when collaborating with peers and advising undergraduates.

\end{document}