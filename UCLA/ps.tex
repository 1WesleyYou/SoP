\documentclass[12pt]{article}
\usepackage[T1]{fontenc}
\usepackage[utf8]{inputenc}
\usepackage{parskip}
\usepackage{setspace}

\setlength{\parskip}{0pt}
\singlespacing
% Linespread command allows you to change line spacing for the entire document
% \linespread{1.18}

% Tweak page margins
\addtolength{\oddsidemargin}{-.875in}
\addtolength{\evensidemargin}{-.875in}
\addtolength{\textwidth}{1.75in}

\addtolength{\topmargin}{-.875in}
\addtolength{\textheight}{1.75in}

\usepackage{natbib}
\usepackage{hyperref}
\usepackage{xcolor}
\usepackage{xspace}
\usepackage{fancyhdr}
\hypersetup{
    colorlinks,
    linkcolor={red!50!black},
    citecolor={blue!50!black},
    urlcolor={blue!80!black}
}

\newcommand{\HRule}{\rule{\linewidth}{0.5mm}}
\newcommand{\Hrule}{\rule{\linewidth}{0.3mm}}

% Project specific macros
\newcommand{\graphite}{GRAPHITE\xspace}
\newcommand{\wave}{WAVE\xspace}

% School specific macros
\newcommand{\schoolShort}{UCLA\xspace}
\newcommand{\school}{University of California, Los Angeles\xspace}
\newcommand{\schoolLong}{University of California, Los Angeles\xspace}

\newcommand{\profOne}{Prof. Ryan Huang\xspace}
\newcommand{\profTwo}{Prof. Xiaonan Huang\xspace}
\newcommand{\profThree}{Prof. Yutong Ban\xspace}

% Creates header for each page
\usepackage{fancyhdr}
\pagestyle{fancy}
\fancyhf{}
\fancyhead[LE,RO]{\header\hskip\linepagesep\vfootline\thepage}
\newskip\linepagesep \linepagesep 5pt\relax
\def\vfootline{%
    \begingroup
    	\rule[-10pt]{0.75pt}{25pt}
    \endgroup
}
\def\header{%
	\begin{minipage}[]{120pt}
		\hfill Yuchen You
    	\par \hfill 				% Formatting boilerplate
    	MSCS, Fall 2026 			% Area, Program, Cycle, Year
    \end{minipage}
}
\fancyhead[RE,LO]{Personal History Statement | \schoolLong}
\renewcommand\headrulewidth{0pt}

\begin{document}

\paragraph{Crossing Academic Cultures and Learning to Ask for Help.}
I am completing two bachelor's degrees in Mechanical Engineering at Shanghai Jiao Tong University and Computer Science at the University of Michigan. Moving between two academic cultures revealed how much ``hidden curriculum'' exists in higher education---unwritten expectations about class participation, office hours, and how to learn from open-ended reading lists. My earlier schooling emphasized high-stakes exams and explicit rubrics, which gave me discipline, but it did not train me to proactively seek informal resources or to ask questions in a second language. A concrete barrier came early in an operating systems course: I over-focused on implementation and underestimated the exam’s emphasis on concepts and reasoning. I treated that mismatch as feedback on my approach, not my ability. I began attending office hours weekly, formed a small study group to compare strategies, and built a routine of reading checkpoints and practice problems. Balancing this adjustment with a second major and research taught me to recover from setbacks by iterating on process and seeking support early---not simply working longer hours.

\paragraph{Inclusive Teaching and Structured Teamwork.}
As a \textbf{teaching assistant} for an \textbf{introductory software engineering course} at SJTU, I saw how team structure can widen or narrow participation: confident students often claimed core coding tasks, while quieter teammates drifted toward documentation or ``safe'' roles. I tried to make contribution pathways more visible and more evenly distributed. In project meetings, I encouraged teams to rotate responsibilities (implementation, testing, debugging, and writing) and used check-in prompts that required every member to explain one design decision or piece of code. In office hours, I intentionally invited questions from students who seemed hesitant to speak up, and I learned to normalize confusion as part of learning rather than a sign of mismatch. Watching students move from ``I'm not a coding person'' to implementing key modules convinced me that inclusion is often a matter of concrete practices---clear expectations, psychological safety, and structured collaboration---not slogans.

\paragraph{Community Service, Digital Inclusion, and Contributions at UCLA.}
I have also volunteered at a community center in Shanghai helping seniors use smartphones for essential tasks such as booking hospital appointments through apps and WeChat mini-programs. Many participants were anxious about touch interfaces, permissions, or making irreversible mistakes, so our work often meant patient, repeated walkthroughs and small confidence-building steps. At UCLA, I would bring the same mindset to research groups, reading groups, and course communities: make expectations explicit, create low-friction pathways to ask questions, and build collaboration norms that help newcomers participate fully. For example, I would like to help organize a newcomer-friendly onboarding process for reading groups---sharing ``how to read systems papers'' templates, hosting low-stakes weekly discussion sessions, and pairing first-time participants with peer buddies. These perspectives align with UCLA's Principles of Community---respect, inclusion, and open exchange across backgrounds---and I hope to contribute through mentoring, peer onboarding, and outreach that expands access to computing.

\end{document}

% That's All Folks.

% Best of luck, you got this! :)