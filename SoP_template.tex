\documentclass{article}
\usepackage[T1]{fontenc}
\usepackage[utf8]{inputenc}
\usepackage{parskip}

\setlength{\parskip}{6pt}
\setlength{\parindent}{10pt}

% Linespread command allows you to change line spacing for the entire document
\linespread{1.18}

% Tweak page margins
\addtolength{\oddsidemargin}{-.875in}
\addtolength{\evensidemargin}{-.875in}
\addtolength{\textwidth}{1.75in}

\addtolength{\topmargin}{-.875in}
\addtolength{\textheight}{1.75in}

\usepackage{natbib}
\usepackage{hyperref}
\usepackage{xcolor}
\usepackage{xspace}
\usepackage{fancyhdr}
\hypersetup{
    colorlinks,
    linkcolor={red!50!black},
    citecolor={blue!50!black},
    urlcolor={blue!80!black}
}

\newcommand{\HRule}{\rule{\linewidth}{0.5mm}}
\newcommand{\Hrule}{\rule{\linewidth}{0.3mm}}

% Project specific macros
\newcommand{\graphite}{GRAPHITE\xspace}
\newcommand{\wave}{WAVE\xspace}

% School specific macros
\newcommand{\schoolShort}{SJTU\xspace}
\newcommand{\school}{Shanghai Jiao Tong University\xspace}
\newcommand{\schoolLong}{Shanghai Jiao Tong University\xspace}

\newcommand{\profOne}{Prof. Ryan Huang\xspace}
\newcommand{\profTwo}{Prof. Xiaonan Huang\xspace}
\newcommand{\profThree}{Prof. Yutong Ban\xspace}

% Creates header for each page
\usepackage{fancyhdr}
\pagestyle{fancy}
\fancyhf{}
\fancyhead[LE,RO]{\header\hskip\linepagesep\vfootline\thepage}
\newskip\linepagesep \linepagesep 5pt\relax
\def\vfootline{%
    \begingroup
    	\rule[-10pt]{0.75pt}{25pt}
    \endgroup
}
\def\header{%
	\begin{minipage}[]{120pt}
		\hfill Yuchen You
    	\par \hfill 				% Formatting boilerplate
    	CS, PhD, Fall 2026 			% Area, Program, Cycle, Year
    \end{minipage}
}
\fancyhead[RE,LO]{Statement of Purpose | \schoolLong}
\renewcommand\headrulewidth{0pt}

\begin{document}

%% Why do you wish to attend graduate school? What would you like to study? Keep it broad, details come-in later
My interest in computer systems began with a fascination for cybersecurity and the low-level logic of how software controls hardware. Although my undergraduate major started in Mechanical Engineering, my core passion has always been in understanding the underlying systems that ensure reliability and efficiency. Through my coursework in Operating Systems and Distributed Systems, I realized that the biggest challenges today are in the intersection of AI and Infrastructure. My goal for a Ph.D. is straightforward: I want to build systems that make machine learning efficient, and conversely, apply machine learning techniques to build more reliable systems.

%% Describe 2-3 past projects that might be relevant to your research interests. (10-12 lines per project)

In my Operating Systems courses, I was not satisfied with just learning how to use standard APIs. I wanted to understand the specific design choices behind them. To do this, I studied classic system literature, ranging from the monolithic design of Unix to the microkernel architecture of Mach, and the virtualization methods in Dune. I also examined how distributed consistency models evolved from GFS to Dynamo. These readings taught me that system design is always about trading off consistency for efficiency. This theoretical foundation gave me the framework to analyze the engineering problems I later faced in practice.

% PROJECT 1: P3 - Distributed Graph Neural Network Training at Scale

My transition from theory to practice happened during my time as the Embedded Lead for our RoboMaster team (National Champions) and my research on soft robotics (ICRA 2025 Best Poster). Working with STM32 microcontrollers and FreeRTOS, I faced the harsh reality of resource constraints. While my teammates focused on high-level control algorithms, I spent my time fixing race conditions, resolving priority inversions, and optimizing memory usage to prevent system crashes. These experiences were pivotal. I realized that even the best control algorithms are useless if the underlying system is unstable or has high latency. This specific pain point motivated me to switch my focus from building robots to building the system software that supports them.

% PROJECT 2: SURGEON - Early-Exit Inference

To solve the reliability issues I encountered, I joined the Order Lab to work on Agentic Distributed System Operations. The problem we tackled was "gray failures"—partial malfunctions in distributed clusters that rule-based tools cannot detect. I built a prototype that uses a Large Language Model (LLM) agent to automatically diagnose and fix these failures. My system integrates Prometheus for monitoring, HAProxy for traffic control, and an LLM for reasoning. A key challenge was that LLMs can be unpredictable. To fix this, I designed a finite-state machine (FSM) to validate the agent’s actions before execution. I tested this system on a ZooKeeper cluster using ChaosBlade for fault injection. The results showed that my agent could automatically recover from complex cascading failures where traditional scripts failed.

% PROJECT 3: Graphite - Distributed Temporal Graph Processing

Parallel to my work on reliability, I also worked on improving efficiency for Large Language Model inference. In the CUDA Proxy Player project, I found that the standard "Eager Launch" mode caused high CPU overhead, especially for dynamic models like Mixture-of-Experts (MoE). I proposed a Hybrid Scheduling method to solve this. I moved stable computations into CUDA Graphs to reduce launch overhead, and I implemented Persistent Kernels on the GPU to handle dynamic tasks (like packing and sorting) without communicating back to the CPU. This approach significantly reduced latency. My implementation achieved a 1.8x speedup in high-frequency scenarios compared to the baseline. This project gave me deep experience in GPU architecture and kernel optimization.

%% Non-research accomplishments (e.g. Grades, Academic Service, Work experience) (10-12 lines)

% Grades

% TA and Academic Service

% Industry

%% Why this school? List professors you would like to work with and why? (10-12 Lines)

%% Summary (3-4 Lines)

% Add some blank space between text and references
% \vspace{0.125in}

% References

% **NOTE**: There are better ways to manage citations in LaTeX, most notably using a bibTeX. I wanted to have greater control on how citations were spaced and formatted and therefore ended up hardcoding them here. Your mileage may wary!

\end{document}

% That's All Folks.

% Best of luck, you got this! :)