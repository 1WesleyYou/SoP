\documentclass[10pt]{article}
\usepackage[T1]{fontenc}
\usepackage[utf8]{inputenc}
\usepackage{parskip}

\setlength{\parskip}{5pt}
% Linespread command allows you to change line spacing for the entire document
\linespread{1.18}

% Tweak page margins
\addtolength{\oddsidemargin}{-.875in}
\addtolength{\evensidemargin}{-.875in}
\addtolength{\textwidth}{1.75in}

\addtolength{\topmargin}{-.875in}
\addtolength{\textheight}{1.75in}

\usepackage{natbib}
\usepackage{hyperref}
\usepackage{xcolor}
\usepackage{xspace}
\usepackage{fancyhdr}
\hypersetup{
    colorlinks,
    linkcolor={red!50!black},
    citecolor={blue!50!black},
    urlcolor={blue!80!black}
}

\newcommand{\HRule}{\rule{\linewidth}{0.5mm}}
\newcommand{\Hrule}{\rule{\linewidth}{0.3mm}}

% Project specific macros
\newcommand{\graphite}{GRAPHITE\xspace}
\newcommand{\wave}{WAVE\xspace}

% School specific macros
\newcommand{\schoolShort}{CMU\xspace}
\newcommand{\school}{Carnegie Mellon University\xspace}
\newcommand{\schoolLong}{Carnegie Mellon University\xspace}

\newcommand{\profOne}{Prof. Ryan Huang\xspace}
\newcommand{\profTwo}{Prof. Xiaonan Huang\xspace}
\newcommand{\profThree}{Prof. Yutong Ban\xspace}

% Creates header for each page
\usepackage{fancyhdr}
\pagestyle{fancy}
\fancyhf{}
\fancyhead[LE,RO]{\header\hskip\linepagesep\vfootline\thepage}
\newskip\linepagesep \linepagesep 5pt\relax
\def\vfootline{%
    \begingroup
    	\rule[-10pt]{0.75pt}{25pt}
    \endgroup
}
\def\header{%
	\begin{minipage}[]{120pt}
		\hfill Yuchen You
    	\par \hfill 				% Formatting boilerplate
    	MSCS, Fall 2026 			% Area, Program, Cycle, Year
    \end{minipage}
}
\fancyhead[RE,LO]{Statement of Purpose | \schoolLong}
\renewcommand\headrulewidth{0pt}

\begin{document}

\textbf{Introduction and Primary Interests.}
I am a senior completing a dual degree in Mechanical Engineering at Shanghai Jiao Tong University and Computer Science at the University of Michigan. Over the past few years, I have gravitated toward \textbf{computer systems and networking}, especially the design of \textbf{distributed systems} and \textbf{infrastructure for data-intensive and ML-driven workloads}. I am applying to the \textbf{Master of Science in Computer Science} program at \school{} to deepen my understanding of how operating systems, distributed protocols, and networked services can be designed to be not only high performance but also \textbf{reliable, observable, and easier to operate}. As more critical applications rely on complex cloud and ML stacks, I hope to build the skills needed to develop and operate such systems in a principled way.

\textbf{Academic Background and Evolving Interests.}
My undergraduate coursework has given me a strong foundation in core computer science while also exposing me to different ways of thinking about complex systems. At Michigan, advanced classes in operating systems, distributed systems, and computer networks trained me to reason about \textbf{concurrency, isolation, and end-to-end protocol design} through substantial implementation projects. Beyond taking these courses, I served as a \textbf{teaching assistant} for an introductory software engineering course, where preparing labs, grading code, and holding office hours forced me to articulate software design concepts clearly and support first-year students as they debugged their projects. At SJTU, mechanical engineering courses in dynamics, control, and mechatronics gave me an appreciation for modeling, feedback, and physical constraints. Initially, I saw these paths as separate, but over time I realized that the ideas I enjoyed most---feedback loops, failure modes, and system behavior under stress---were shared across both disciplines. This realization, together with volunteer work in a local community program where I sat beside older residents and helped them navigate smartphone-based services step by step, shifted my focus from individual devices and robots to the large-scale software systems that now underpin critical services and taught me to be patient, empathetic, and proactive in seeking and offering help. Navigating two universities with different expectations around independent study, class participation, and office hours has likewise taught me to seek out resources proactively rather than treating syllabi as fixed checklists.

\textbf{Systems and Infrastructure Projects.}
My strongest preparation for graduate-level systems work has come from multi-semester projects where I helped build and evaluate end-to-end systems. At the University of Michigan's \textbf{OrderLab}, advised by \profOne, I work on \textbf{automating operations} for a ZooKeeper-based distributed coordination service. Rather than designing new detectors, our project assumes that anomaly signals already exist and asks a pragmatic question: given \textbf{noisy metrics and coarse alerts}, can an automated operator select mitigation actions that keep client-facing latency within SLOs? I helped design and implement an experimentation framework around a ZooKeeper cluster: \textbf{Prometheus}-based metrics and alerts, \textbf{HAProxy}-based traffic shaping, a small typed library of mitigation actions (e.g., throttling, rerouting, I/O limiting) with safety guards, and synthetic workloads that induce overload and gray failures. A simple threshold-based controller could prevent collapse but often overreacted to spikes and underreacted to slow degradations, leaving long periods of elevated tail latency. Building on this baseline, I implemented an \textbf{agentic layer} where a planner periodically reads metrics and finite-state summaries and proposes short sequences of mitigations from our constrained library. Early negative results, where richer control surfaces led to oscillations and over-throttling, taught me how easily such systems can become unstable; tightening the observation interface, simplifying the action space, and validating plans via replay made the system safer and more effective. This project has shaped how I think about \textbf{operational robustness} and the importance of good telemetry and carefully designed control surfaces in complex distributed systems.

In an \textbf{Advanced Operating Systems} course project, I explored another side of infrastructure by building \textbf{CUDA Proxy Player}, a runtime for \textbf{GPU-based inference workloads}. Our hypothesis was that for certain \textbf{Mixture-of-Experts} models, \textbf{host-side orchestration and kernel launches} rather than raw FLOPs dominate end-to-end latency. Working in a small team, I co-designed a multi-path runtime that routes requests by size and shape to an eager path, a \textbf{persistent-kernel} path, or a \textbf{CUDA Graph} path, and ``glues'' these execution flows together by managing graph capture, replay, and kernel scheduling. I implemented much of the \textbf{benchmarking framework} and used it to compare these paths under synthetic traffic. Our experiments showed that in a launch-bound regime, the graph-based path can modestly reduce latency compared to a naive eager baseline but also highlighted trade-offs: persistent kernels that keep workers and buffers resident can shrink VRAM headroom for co-located jobs and worsen tail latency in multi-tenant settings. This experience broadened my view from ``finding the fastest kernel'' to thinking about \textbf{resource sharing, observability, and admission control} in GPU-centric systems.

\textbf{Embedded Robotics and an Unusual Path into Systems.}
Before moving into large-scale software infrastructure, I spent several years in \textbf{embedded systems and robotics}. At UMich's \textbf{HDRLab} with \profTwo, I worked on an \textbf{origami-inspired modular robotic arm}, writing low-level firmware and higher-level control software that connected MCUs, sensors, and actuators \textbf{over buses such as CAN and I\textsuperscript{2}C} and executed model-based dynamics under tight power, bandwidth, and timing constraints. Failures rarely appeared as simple crashes; instead, they emerged as \textbf{noisy, delayed, or corrupted sensor data} that slowly pushed the arm toward unsafe regimes. Adding lightweight monitoring and metrics---filters, sanity checks, and degraded-but-safe fallback modes when data quality dropped---significantly improved robustness and gave me an intuition for concepts like \textbf{observability}, \textbf{fault containment}, and \textbf{graceful degradation} under strict resource budgets. This somewhat unusual path into systems, across engineering and CS, has made me comfortable working at boundaries between hardware and software, control and computation, and different academic fields.

\textbf{How I Work and What I Hope to Gain from Graduate Study.}
Across these projects, I have found that I work best by \textbf{reasoning from measurements}: instrumenting systems, collecting traces, and using them to refine models of system behavior. I enjoy \textbf{end-to-end engineering}, from reading existing codebases and designing experiments to implementing mechanisms and writing careful evaluations, and I have learned to treat negative results as chances to iterate rather than failures. I also value collaboration in small teams. In graduate study, I hope to deepen my understanding of \textbf{distributed systems} and related areas in \textbf{networking, storage, databases, and ML systems}, and to gain more exposure to both \textbf{formal reasoning about correctness} and working with large production-like codebases.

\textbf{Fit with the CMU MSCS Program.}
Two threads have emerged from my undergraduate work: building telemetry-driven control loops for distributed services, and designing runtimes for ML inference workloads on GPUs. I see the CMU MSCS program as the right place to push both threads further. Courses such as \textbf{15-712 (Advanced and Distributed Operating Systems)} and \textbf{15-749 (Engineering Distributed Systems)} would let me revisit these kinds of systems in a more principled way, through readings and substantial, project-based work. I am also excited by the broader SCS systems and data infrastructure community.

In the MSCS program, I hope to turn my project experience into deeper, more transferable expertise. I would like to build on my ZooKeeper mitigation work by studying more formal connections between noisy telemetry, control policies, and end-to-end SLOs, and to complement my CUDA runtime experiments with coursework at the systems/ML boundary on scheduling, isolation, and multi-tenant GPU efficiency. This preparation would position me to work on the design and operation of \textbf{large-scale distributed and ML-serving systems} in environments where reliability and performance matter, whether in industry research labs or, potentially, in a future doctoral program.

I am not submitting GRE General Test scores for this application. Within the constraints of this application cycle, I chose to prioritize advanced coursework and systems research over preparing for an additional standardized exam, and I hope that my record in mathematics and upper-level computer science, together with my research, provides a clearer demonstration of my analytical preparation for the MSCS curriculum and my readiness to contribute to the program.

\end{document}