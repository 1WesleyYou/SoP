\documentclass{article}
\usepackage[T1]{fontenc}
\usepackage[utf8]{inputenc}
\usepackage{parskip}

\setlength{\parskip}{8pt}
% Linespread command allows you to change line spacing for the entire document
\linespread{1.18}

% Tweak page margins
\addtolength{\oddsidemargin}{-.875in}
\addtolength{\evensidemargin}{-.875in}
\addtolength{\textwidth}{1.75in}

\addtolength{\topmargin}{-.875in}
\addtolength{\textheight}{1.75in}

\usepackage{natbib}
\usepackage{hyperref}
\usepackage{xcolor}
\usepackage{xspace}
\usepackage{fancyhdr}
\hypersetup{
    colorlinks,
    linkcolor={red!50!black},
    citecolor={blue!50!black},
    urlcolor={blue!80!black}
}

\newcommand{\HRule}{\rule{\linewidth}{0.5mm}}
\newcommand{\Hrule}{\rule{\linewidth}{0.3mm}}

% Project specific macros
\newcommand{\graphite}{GRAPHITE\xspace}
\newcommand{\wave}{WAVE\xspace}

% School specific macros
\newcommand{\schoolShort}{Princeton\xspace}
\newcommand{\school}{Princeton University\xspace}
\newcommand{\schoolLong}{Princeton University\xspace}

\newcommand{\profOne}{Prof. Ryan Huang\xspace}
\newcommand{\profTwo}{Prof. Xiaonan Huang\xspace}
\newcommand{\profThree}{Prof. Yutong Ban\xspace}

% Creates header for each page
\usepackage{fancyhdr}
\pagestyle{fancy}
\fancyhf{}
\fancyhead[LE,RO]{\header\hskip\linepagesep\vfootline\thepage}
\newskip\linepagesep \linepagesep 5pt\relax
\def\vfootline{%
    \begingroup
    	\rule[-10pt]{0.75pt}{25pt}
    \endgroup
}
\def\header{%
	\begin{minipage}[]{120pt}
		\hfill Yuchen You
    	\par \hfill 				% Formatting boilerplate
    	MSCS, Fall 2026 			% Area, Program, Cycle, Year
    \end{minipage}
}
\fancyhead[RE,LO]{Personal History Statement | \schoolLong}
\renewcommand\headrulewidth{0pt}

\begin{document}

I am completing two bachelor’s degrees---Mechanical Engineering at Shanghai Jiao Tong University and Computer Science at the University of Michigan---and the hardest part was not workload, but switching academic cultures. My earlier schooling rewarded high-stakes exams and explicit rubrics. At Michigan, I encountered a different set of expectations: open-ended reading lists, unspoken norms around office hours, and the challenge of asking precise questions in a second language. That ``hidden curriculum'' became concrete in an operating systems course, where I over-optimized for implementation and underestimated how much the course valued reasoning about concurrency and failure modes.

Instead of treating that setback as a verdict, I treated it as feedback and built a repeatable adaptation loop. Before each lecture, I wrote down what I did \emph{not} understand from the readings; after each lecture, I converted those gaps into targeted questions for office hours. Repeating the cycle---try, get stuck, ask early, iterate---taught me that academic growth is often social: it depends on making uncertainty discussable and turning tacit expectations into shared tools.

Those lessons shape how I hope to contribute at Princeton. In the residential scholarly community, I would help lower the activation energy for joining academic conversations: newcomer-friendly reading discussions, lightweight ``how to read systems papers'' templates, and informal peer Q\&A sessions that normalize early questions. More broadly, I want to be the person who notices when a norm is implicit and helps make it visible---so that students from different educational systems and language backgrounds can participate fully, sooner.

\end{document}

% That's All Folks.

% Best of luck, you got this! :)