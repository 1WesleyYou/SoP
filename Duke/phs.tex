\documentclass[12pt]{article}
\usepackage[T1]{fontenc}
\usepackage[utf8]{inputenc}
\usepackage{setspace}
\usepackage{parskip}

\singlespacing
\setlength{\parskip}{8pt}
% Linespread command allows you to change line spacing for the entire document
% \linespread{1.18}

% Tweak page margins
\addtolength{\oddsidemargin}{-.875in}
\addtolength{\evensidemargin}{-.875in}
\addtolength{\textwidth}{1.75in}

\addtolength{\topmargin}{-.875in}
\addtolength{\textheight}{1.75in}

\usepackage{natbib}
\usepackage{hyperref}
\usepackage{xcolor}
\usepackage{xspace}
\usepackage{fancyhdr}
\hypersetup{
    colorlinks,
    linkcolor={red!50!black},
    citecolor={blue!50!black},
    urlcolor={blue!80!black}
}

\newcommand{\HRule}{\rule{\linewidth}{0.5mm}}
\newcommand{\Hrule}{\rule{\linewidth}{0.3mm}}

% Project specific macros
\newcommand{\graphite}{GRAPHITE\xspace}
\newcommand{\wave}{WAVE\xspace}

% School specific macros
\newcommand{\schoolShort}{Duke\xspace}
\newcommand{\school}{Duke University\xspace}
\newcommand{\schoolLong}{Duke University\xspace}

\newcommand{\profOne}{Prof. Ryan Huang\xspace}
\newcommand{\profTwo}{Prof. Xiaonan Huang\xspace}
\newcommand{\profThree}{Prof. Yutong Ban\xspace}

% Creates header for each page
\usepackage{fancyhdr}
\pagestyle{fancy}
\fancyhf{}
\fancyhead[LE,RO]{\header\hskip\linepagesep\vfootline\thepage}
\newskip\linepagesep \linepagesep 5pt\relax
\def\vfootline{%
    \begingroup
    	\rule[-10pt]{0.75pt}{25pt}
    \endgroup
}
\def\header{%
	\begin{minipage}[]{120pt}
		\hfill Yuchen You
    	\par \hfill 				% Formatting boilerplate
    	CS, PhD, Fall 2026 			% Area, Program, Cycle, Year
    \end{minipage}
}
\fancyhead[RE,LO]{Life Experiences Statement | \schoolLong}
\renewcommand\headrulewidth{0pt}

\begin{document}

\textbf{Life Experiences Statement}

I grew up in China and later spent the final two years of my undergraduate study at the University
of Michigan. Moving into a new language and academic culture was exciting, but it also made the
everyday parts of university life feel surprisingly hard at first---not just classes, but how to
join conversations, how to form friendships, and how to feel that I truly belonged in the campus
community.

In the beginning, I often felt out of sync with the pace and style of discussion. I hesitated to
speak up, worried that my phrasing or tone might sound awkward, and I sometimes kept questions to
myself even when I was confused. Over time, I learned that adjustment is not about becoming
someone else, but about building the confidence to participate as myself. I started taking small
steps: joining study groups, staying after class to talk with classmates, and saying yes to
invitations even when I felt nervous. Gradually, those small steps turned into friendships. I
became close with students from many countries, and these relationships have been one of the most
meaningful parts of my experience in the U.S. They broadened my perspective on how different
people learn, communicate, and support each other, and they taught me how much warmth and
understanding can exist across cultures when people are willing to listen.

Because I know what it feels like to struggle silently at the edge of a new environment, I want
to help future international students navigate cultural and language barriers more smoothly. At
Duke, I hope to be a welcoming peer---someone who reaches out to newcomers, helps connect them to
friends and study partners, and shares practical strategies for participating in classes and
campus life with confidence. I also hope to contribute to a friendly community where students
exchange perspectives with curiosity and respect, and where differences become a shared resource
for learning.

I would also be excited to serve as a teaching assistant at Duke. I enjoy helping others build
intuition and confidence, and I want to support students who may hesitate to ask questions because
they feel behind, shy, or unsure in English. My goal is to help create classrooms and discussion
spaces where it is normal to ask ``basic'' questions, where explanations are patient and clear, and
where students feel encouraged to keep going when the learning curve is steep.

\end{document}