\documentclass{article}
\usepackage[T1]{fontenc}
\usepackage[utf8]{inputenc}
\usepackage{parskip}

\setlength{\parskip}{5pt}
% \setlength{\parindent}{10pt}

%  TODO: 注意这里的用词要符合论文风格, 专业名词这些

% Linespread command allows you to change line spacing for the entire document
\linespread{1.18}

% Tweak page margins
\addtolength{\oddsidemargin}{-.875in}
\addtolength{\evensidemargin}{-.875in}
\addtolength{\textwidth}{1.75in}

\addtolength{\topmargin}{-.875in}
\addtolength{\textheight}{1.75in}

\usepackage{natbib}
\usepackage{hyperref}
\usepackage{xcolor}
\usepackage{xspace}
\usepackage{fancyhdr}
\hypersetup{
    colorlinks,
    linkcolor={red!50!black},
    citecolor={blue!50!black},
    urlcolor={blue!80!black}
}

\newcommand{\HRule}{\rule{\linewidth}{0.5mm}}
\newcommand{\Hrule}{\rule{\linewidth}{0.3mm}}

% Project specific macros
\newcommand{\graphite}{GRAPHITE\xspace}
\newcommand{\wave}{WAVE\xspace}

% School specific macros
\newcommand{\schoolShort}{Purdue\xspace}
\newcommand{\school}{Purdue University\xspace}
\newcommand{\schoolLong}{Purdue University West Lafayette\xspace}

\newcommand{\profOne}{Prof. Ryan Huang\xspace}
\newcommand{\profTwo}{Prof. Xiaonan Huang\xspace}
\newcommand{\profThree}{Prof. Yutong Ban\xspace}

% Creates header for each page
\usepackage{fancyhdr}
\pagestyle{fancy}
\fancyhf{}
\fancyhead[LE,RO]{\header\hskip\linepagesep\vfootline\thepage}
\newskip\linepagesep \linepagesep 5pt\relax
\def\vfootline{%
    \begingroup
    	\rule[-10pt]{0.75pt}{25pt}
    \endgroup
}
\def\header{%
	\begin{minipage}[]{120pt}
		\hfill Yuchen You
    	\par \hfill 				% Formatting boilerplate
    	CS, PhD, Fall 2026 			% Area, Program, Cycle, Year
    \end{minipage}
}
\fancyhead[RE,LO]{Personal History Statement | \schoolLong}
\renewcommand\headrulewidth{0pt}

\begin{document}

\textbf{Personal Education History and Challenges.}
I grew up in Zhejiang, China in a family that cared more about curiosity than competition. At school, however, success was often defined almost entirely by exam scores. I did well in that system, but I also saw classmates who struggled quietly when they missed a concept early on and then felt that the door to science or engineering had closed. I often sat with them after class to go over homework step by step. Those small conversations were the first time I realized how much a little extra explanation and encouragement can change someone's sense of belonging in a technical field.

In college, I chose a path that pulled me between two different higher education systems: mechanical engineering at Shanghai Jiao Tong University and computer science at the University of Michigan. This meant two sets of degree requirements, calendars, and advising structures. In my first two years, my schedule was packed with mandatory courses, and it was hard to reserve time for early research or to follow a standard ``CS major'' plan. I often felt behind when I compared myself with peers who could focus on one curriculum. To cope, I treated my coursework like a dependency graph, planning each term so that I could open space for systems and networking classes as soon as possible.

Moving from Shanghai to Ann Arbor added another set of challenges. I had to adjust to courses taught in English, different expectations about class participation, and norms around using office hours. At first, I misunderstood some of these norms and tried to handle everything on my own. In Advanced Operating Systems, for example, I focused almost entirely on coding the projects and did not spend enough time reading the papers or asking questions. My midterm performance was much worse than I expected. After that, I forced myself to visit office hours, admit what I did not understand, and ask the professor how they expected us to study. That experience taught me that success in this environment depends not only on effort, but also on knowing how to use resources that are not always obvious to students who did not grow up in the same system. It was my first personal encounter with what I now think of as the ``hidden curriculum'' of research universities.

\textbf{Community Engagement and the Digital Divide.}
While living in Shanghai, I gradually noticed that many older neighbors around me were both lonely and disconnected from the way services were moving online. Everyday tasks such as booking hospital appointments, paying bills, or talking with distant family had shifted to smartphone apps and mini–programs. For many of these older adults, the phone was not a tool of convenience but a reminder that the world was moving on without them.

In response, I worked with roommates to organize a small volunteer visit at a nearby senior activity center. During that visit, each of us sat next to one older adult at a time, helped them unlock their phone, open the right WeChat mini–program, and complete a basic task such as reserving a hospital appointment, often repeating the same steps patiently. I also made sure that we left time simply to talk, listen to their stories, and treat the visit as more than a technical lesson. Some participants apologized for being ``slow'' or said that technology was ``only for young people,'' but even within that short session, when they succeeded in booking an appointment or sending a photo to a grandchild on their own, I could see their confidence return. Organizing and leading this visit changed how I think about technology and fairness: the digital divide is not only about networks or hardware, but also about design, patience, and whether someone is willing to sit beside you while you learn.

\textbf{Mentoring and Inclusive Teaching.}
At Shanghai Jiao Tong University, I later became a teaching assistant for ENGR~1000J, an introductory course where first-year students build a small web-based game in teams. Many students entered the course with little or no programming background and were hesitant to touch the code at all. In early design meetings, I saw quieter or shy students remain silent even when they had interesting ideas, while a few more confident classmates quickly took over most of the implementation work. As a result, some of the most creative design directions never reached the team, and several groups moved more slowly than they needed to because only a small subset of voices shaped both the architecture and the code.

As a TA, I tried to change this dynamic in small, concrete ways. When I met with teams, I encouraged them to break the project into smaller features and to let each member own at least one piece of functionality, no matter how simple. I asked more experienced students to explain their design decisions out loud and rotate roles. For students who were hesitant to write code, I sat with them one-on-one to implement their first feature, line by line, until they felt comfortable pushing a change. By the end of the term, almost every student had something in the project they could point to and say, ``I built that.'' For me, the most meaningful feedback was when students who had doubted their abilities told me that this was the first time they felt included in a technical project rather than just observing it from the edge.

These experiences, as a student and as a TA, have shaped how I see education and community. I have been lucky to study in strong institutions in both China and the United States, but I have also seen how easy it is for capable people to be left out because they do not yet know the unwritten rules or do not see themselves reflected in the most confident voices in the room.

\textbf{Future Contributions at Purdue.}
At Purdue, I hope to contribute not only as a researcher in computer systems, but also as someone who helps others navigate demanding environments. Drawing on my experience moving between two university systems, I plan to \textbf{serve as a teaching assistant} who makes expectations explicit and shares the ``hidden curriculum'' around office hours, large projects, and finding research opportunities, so that students who are new to systems, to computer science, or to the U.S.\ academic system do not have to figure everything out alone. I also hope to take part in cross-cultural student communities and to help build project teams where differences in language, culture, and background are treated as strengths rather than barriers. By continuing to challenge myself in advanced coursework and research---and being honest about the struggles behind each step---I want to help create an environment where students from many backgrounds can persist, ask for help without shame, and feel that they truly belong at Purdue.

\end{document}
