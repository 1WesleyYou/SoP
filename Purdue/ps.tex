\documentclass{article}
\usepackage[T1]{fontenc}
\usepackage[utf8]{inputenc}
\usepackage{parskip}

\setlength{\parskip}{10pt}
% \setlength{\parindent}{10pt}

%  TODO: 注意这里的用词要符合论文风格, 专业名词这些

% Linespread command allows you to change line spacing for the entire document
\linespread{1.18}

% Tweak page margins
\addtolength{\oddsidemargin}{-.875in}
\addtolength{\evensidemargin}{-.875in}
\addtolength{\textwidth}{1.75in}

\addtolength{\topmargin}{-.875in}
\addtolength{\textheight}{1.75in}

\usepackage{natbib}
\usepackage{hyperref}
\usepackage{xcolor}
\usepackage{xspace}
\usepackage{fancyhdr}
\hypersetup{
    colorlinks,
    linkcolor={red!50!black},
    citecolor={blue!50!black},
    urlcolor={blue!80!black}
}

\newcommand{\HRule}{\rule{\linewidth}{0.5mm}}
\newcommand{\Hrule}{\rule{\linewidth}{0.3mm}}

% Project specific macros
\newcommand{\graphite}{GRAPHITE\xspace}
\newcommand{\wave}{WAVE\xspace}

% School specific macros
\newcommand{\schoolShort}{SJTU\xspace}
\newcommand{\school}{Shanghai Jiao Tong University\xspace}
\newcommand{\schoolLong}{Shanghai Jiao Tong University\xspace}

\newcommand{\profOne}{Prof. Ryan Huang\xspace}
\newcommand{\profTwo}{Prof. Xiaonan Huang\xspace}
\newcommand{\profThree}{Prof. Yutong Ban\xspace}

% Creates header for each page
\usepackage{fancyhdr}
\pagestyle{fancy}
\fancyhf{}
\fancyhead[LE,RO]{\header\hskip\linepagesep\vfootline\thepage}
\newskip\linepagesep \linepagesep 5pt\relax
\def\vfootline{%
    \begingroup
    	\rule[-10pt]{0.75pt}{25pt}
    \endgroup
}
\def\header{%
	\begin{minipage}[]{120pt}
		\hfill Yuchen You
    	\par \hfill 				% Formatting boilerplate
    	CS, PhD, Fall 2026 			% Area, Program, Cycle, Year
    \end{minipage}
}
\fancyhead[RE,LO]{Personal History Statement | \schoolLong}
\renewcommand\headrulewidth{0pt}

\begin{document}

I grew up in a family that consistently encouraged curiosity, learning, and trying new things rather than focusing only on grades or a single “correct” path. From primary school through middle school, I was fortunate to receive a solid education and broad exposure to science, humanities, and the arts, which helped shape not only my academic interests but also my values and worldview. Early on, I was taught to treat people with respect regardless of their background, to stay optimistic in the face of difficulties, and to be willing to help classmates whenever I could. These experiences made me naturally inclined to support people around me, to listen before judging, and to create an environment where everyone feels included and able to participate.

When I entered university, these values continued to shape how I navigated new environments. At Shanghai Jiao Tong University, I first majored in Mechanical Engineering and later joined a dual-degree program with the University of Michigan to study Computer Science. Switching not only disciplines but also education systems was both exciting and challenging: I had to adapt to different teaching styles, project expectations, and academic cultures, all while studying in a second language. I was fortunate to receive strong institutional support and to have the confidence to seek out new opportunities, but I also saw how some classmates-especially those less familiar with the language, resources, or informal ``rules'' of the system-could feel lost or excluded. These experiences made me more aware that access to opportunity is often uneven, even among students sitting in the same classroom, and strengthened my desire to make academic spaces more welcoming and navigable for others.

These concerns became concrete for me when I served as a teaching assistant for ENGR 1000J, an introductory software engineering course for freshman at Shanghai Jiao Tong University. Many of my students had never built a real project before, and our final assignment asked them to design and implement a web-based game in teams. Early in the semester, I noticed a familiar pattern: a few confident students would quickly take over the ``core'' tasks—game mechanics, backend logic—while others quietly drifted to the margins, doing only documentation or minor UI tweaks. In labs and office hours, I made a deliberate effort to break this pattern: I worked with each team to decompose the project into smaller, meaningful components, encouraged them to rotate roles, and asked the more experienced members to explain their design choices rather than simply ``doing it for everyone.'' I spent extra time with students who were hesitant to code, helping them implement their first interactive features and debug their own work instead of handing them a finished solution. By the end of the course, many teams reported that every member had at least one cool feature they were proud to ``own,'' and several students told me that this was the first time they felt genuinely included in a technical project. Seeing first-year students gain confidence, find their voice in team discussions, and experience the satisfaction of making their ideas work on screen reinforced my belief that thoughtful mentoring and inclusive project design can significantly lower the barriers to engaging with engineering.

Outside the classroom, I tried to carry the same mindset into community work. Through a volunteer program in Shanghai, I regularly visited a senior activity center in Putuo District, where many of the residents lived alone and had limited experience with smartphones. Our main task was to teach them how to use mobile apps and mini-programs to make hospital appointments, check registration numbers, and navigate online payment systems. At first, even basic actions—such as distinguishing between a pop-up ad and a system notification, or understanding why a verification code timeouts—felt overwhelming for many of them. Instead of simply “doing it for them,” I sat down one-on-one, walked through each step slowly, and encouraged them to try on their own while I stayed nearby. Between technical explanations, we also spent time just talking: about their families, their worries about getting timely medical care, and how rapidly the city had changed around them. Over time, I saw several seniors move from frustration to a quiet sense of independence when they could successfully book an appointment without asking their children or neighbors for help. This experience made the idea of the “digital divide” very tangible to me—access to technology is not only a matter of infrastructure, but also of patient guidance, trust, and emotional support—and deepened my conviction that technical systems should be designed and taught in ways that include, rather than unintentionally exclude, vulnerable groups.

One important test of my persistence came in EECS 482 (Introduction to Operating Systems). I initially treated it like any other demanding course, completing the projects without fully understanding the underlying ideas, and as a result I scored well below the class average on the first midterm. For a while I felt genuinely discouraged and wondered if I simply was not suited for systems. Instead of quietly giving up, I met with the instructor, Prof. Ryan Huang, who encouraged me to treat the course as research training: study a real OS textbook, revisit each concept carefully, and use the projects as a sandbox to ask “what if” questions and try small extensions. Following his advice, I changed my approach—systematically reviewing the material, designing and reasoning about potential optimizations, and regularly going to office hours to discuss design trade-offs rather than just fix bugs. My performance in the second half of the course improved sharply, and I finished with a final exam score well above the average. More importantly, I learned that when I hit a wall, I can respond not by lowering my ambitions, but by redesigning my strategy and seeking out help, a mindset that I will carry into graduate-level research.

\end{document}
